\usetikzlibrary{shapes.geometric,matrix, calc, patterns, angles, quotes, babel, snakes, 3d, intersections,hobby,positioning, fadings,decorations.pathreplacing,decorations.pathmorphing,er,calc,backgrounds,shapes,arrows,arrows.meta, shadows}

%\usetikzlibrary{babel}%[spanish]
%\usepackage{circuitikz} %simbolos de circuitos
%\ctikzset{bipoles/length=.4cm}

%Global options:
\tikzset{color=blue,fill=blue!50,text=black!80}
\tikzset{water/.style={draw, blue, opacity=0.4, outer color=blue, inner color=blue!30},
solid/.style={draw,black!80,opacity=0.8, outer color=black!30, inner color=black!10},
ground/.style={fill,pattern=north east lines,draw=none,minimum width=0.3,minimum height=0.},
surface/.style={fill,pattern=north east lines,minimum width=0.3,minimum height=0.5},
atmosphere/.style={draw, blue, opacity=0.5, outer color=blue!10, inner color=blue!50},
sun/.style={draw, yellow, opacity=0.5, outer color=yellow!20, inner color=orange!10},
cloud/.style={draw, gray!10, opacity=0.5, outer color=white, inner color=gray!40},
rock/.style={fill,pattern=dots},
building/.style={fill,pattern=bricks, pattern color=black!70}
}
\tikzset{earth/.style={draw, blue, opacity=0.4, outer color=blue, inner color=green!70},
water/.style={draw, blue, opacity=0.4, outer color=blue, inner color=blue!30},
solid/.style={draw,black!80,opacity=0.8, outer color=black!30, inner color=black!10},
ground/.style={fill,pattern=north east lines,draw=none},%,minimum width=0.3,minimum height=0.5
}
\tikzset{every pipe/.code={%
  \pgfmathloop
  \ifnum\pgfmathcounter>20
  \else
    \pgfmathsetmacro\i{\pgfmathcounter}%
    \pgfmathsetmacro\j{80-\pgfmathcounter*3}%
    \tikzset{preaction/.expanded={line width=(\pipewidth)*(1-\i*\i/500),
        draw=black!50!white!\j!white}}%
  \repeatpgfmathloop
},
every flange/.style 2 args={insert path={
  let \p1=(#1), \p2=(#2),\n1={atan2(\y2-\y1, \x2-\x1)} in
    [shift={(\p1)}, rotate=\n1, pipe width={(\pipewidth)*4/3}, every pipe] 
      (0.01,0) -- (0.2,0)
}},
flanges/.style={postaction={%
  /utils/exec={
    \@for\pos:=#1\do{%
      \tikzset{decoration/.expanded={markings,
        mark=at position \pos\space with {
          \noexpand\path [pipe width=\pipewidth*4/3, every pipe] (0.01,0) -- (0.2,0);
    }}}}}, 
  decorate}},
pipe width/.code=\pgfmathsetlengthmacro\pipewidth{#1},
pipes/.style={
  every pipe,
  postaction={decoration={show path construction,
    lineto code={
      \path [every flange={\tikzinputsegmentfirst}{\tikzinputsegmentlast}];
      \path [every flange={\tikzinputsegmentlast}{\tikzinputsegmentfirst}];
    },
    curveto code={
      \path [every flange={\tikzinputsegmentfirst}{\tikzinputsegmentsupporta}];
      \path [every flange={\tikzinputsegmentlast}{\tikzinputsegmentsupportb}];
    }
}, decorate}},
pipe/.style={
  every pipe,
  flanges={0, 1}
}}
\tikzstyle{myarrows}=[line width=1mm,draw=blue,-triangle 45,postaction={draw, line width=3mm, shorten >=4mm, -}]
       
\tikzoption{canvas is plane}[]{\@setOxy#1}
  \def\@setOxy O(#1,#2,#3)x(#4,#5,#6)y(#7,#8,#9)%
    {\def\tikz@plane@origin{\pgfpointxyz{#1}{#2}{#3}}%
     \def\tikz@plane@x{\pgfpointxyz{#4}{#5}{#6}}%
     \def\tikz@plane@y{\pgfpointxyz{#7}{#8}{#9}}%
     \tikz@canvas@is@plane
  }

%Linea desprolija tipo a mano
\tikzstyle{randomLine}=[decorate,decoration={random steps,amplitude=0.5pt}]

%\newcommand\esymbol[1]{\begin{circuitikz}
%\ctikzset{bipoles/length=.6cm}
%\draw (0,0) to [#1] (.75,0); \end{circuitikz}}



%Ejes 2D:
\newcommand\axisDD[2]{
      \draw[->,xshift=#2,thick](0,0)--(#1,0) node[anchor=north]{$\hat{x}$};
      \draw[->,xshift=#2,thick](0,0)--(0,#1) node[anchor=east]{$\hat{y}$};
      \fill[black,xshift=#2](0,0)circle[radius=0.075]node[black,anchor=north]{$\mathbf{o}$};
}
%Ejes 3D:

%Ejes 3D:
\newcommand\axisDDD[3]{
      \draw[-stealth,very thick,black!80,xshift=#2,yshift=#3](0,0,0)--(#1,0,0) node[anchor=north]{$\hat{x}$};
      \draw[-stealth,very thick,black!80,xshift=#2,yshift=#3](0,0,0)--(0,0,#1) node[anchor=east]{$\hat{y}$};
      \draw[-stealth,very thick,black!80,xshift=#2,yshift=#3](0,0,0)--(0,#1,0) node[anchor=east] {$\hat{z}$};
      \fill[black!80,xshift=#2,yshift=#3](0,0)circle[radius=0.05];%node[black,anchor=north]{$\mathbf{o}$};
}
%\newcommand\axisDDD[2]{
%      \draw[->,xshift=#2,thick](0,0,0)--(#1,0,0) node[anchor=north]{$\hat{x}$};
%      \draw[->,xshift=#2,thick](0,0,0)--(0,#1,0) node[anchor=east] {$\hat{y}$};
%      \draw[->,xshift=#2,thick](0,0,0)--(0,0,#1) node[anchor=north]{$\hat{z}$};
%      \fill[black,xshift=#2](0,0)circle[radius=0.075]node[black,anchor=north]{$\mathbf{o}$};
%}
%Ejemplo de cuerpo rígido
\newcommand\cuerporigidoA{\draw[fill=black!10,thick] plot[smooth cycle , tension=.7,fill=black!10]coordinates{
%(-0.72, -0.03)
(-1.05, 0.)(-1.32, 0.29) (-1.41, 0.85) (-1.46, 1.4) (-1.24, 1.6) (-0.81, 1.56) (-0.39, 1.28) (0.15, 1.21) (0.39, 0.54)(-0.07, -0.16) (-0.46, -0.2) };}


\newcommand\cuerporigidoB{\draw[fill=black!10] plot[smooth cycle , tension=.8,fill=black!10]coordinates{(-0.59, -0.08)(-1.08, 0.)(-1.39, 0.1)(-1.63, 0.59)(-1.66, 1.1)(-1.53, 1.54)(-1.41, 2.0)(-1.13, 2.31)(-0.65, 2.37)(-0.27, 2.16)(-0.1, 1.7)(-0.1, 1.1)(-0.11, 0.52)
(-0.19, 0.12)
};}


\newcommand\padDardos[2]{
      \draw[black!50, very thin] (#1,#2)	circle [radius=.2];
      \draw[black!50, thin](#1,#2) 		circle [radius=.3];
      \draw[black!50, ](#1,#2) 			circle [radius=.4];
      \draw[black!50, thick](#1,#2) 		circle [radius=.5];
      \draw[black!50,      thick](#1,#2)  circle [radius=.6];
      \draw[black!50, very thick](#1,#2)  circle [radius=.7];
      \fill[red!80] (#1,#2) circle[radius=0.1];
}

\newcommand\CubeDDD[2]{
        \draw[#2](-#1,-#1,-#1)--(-#1, #1,-#1)--(-#1,#1,#1)--(-#1,-#1,#1)--(-#1,-#1,-#1);
        \draw[#2](-#1,-#1,-#1)--( #1,-#1,-#1)--(#1,-#1,#1)--(-#1,-#1,#1)--(-#1,-#1,-#1);
        \draw[#2](-#1,-#1,-#1)--(#1,-#1,-#1)--(#1,#1,-#1)--(-#1,#1,-#1)--(-#1,-#1,-#1);
        \draw[#2](#1,#1,#1)--(#1,-#1,#1)--(#1,-#1,-#1)--(#1,#1,-#1)--(#1,#1,#1);
        \draw[#2](#1,#1,#1)--(-#1,#1,#1)--(-#1,#1,-#1)--(#1,#1,-#1)--(#1,#1,#1);
        \draw[#2](#1,#1,#1)--(-#1,#1,#1)--(-#1,-#1,#1)--(#1,-#1,#1)--(#1,#1,#1);
%		% inside frame - this shouldn't be necessary...
%         \draw(-#1,-#1,-#1)--(-#1,#1,-#1)--(-#1,#1,#1)--(-#1,-#1,#1)--(-#1,-#1,-#1);
%         \draw(-#1,-#1,-#1)--(#1,-#1,-#1)--(#1,-#1,#1)--(-#1,-#1,#1)--(-#1,-#1,-#1);
%         \draw(-#1,-#1,-#1)--(#1,-#1,-#1)--(#1,#1,-#1)--(-#1,#1,-#1)--(-#1,-#1,-#1);
}
       
\newcommand\roughLine[4]{%xstart,xend,samples,amplitude
			\draw plot[domain=#1:#2, samples=#3] ({\x},{rand*#4});
}

\newcommand\irregularcircle[2]{% radius, irregularity
  \pgfextra {\pgfmathsetmacro\len{(#1)+rand*(#2)}}
  +(0:\len pt)
  \foreach \a in {10,20,...,350}{
    \pgfextra {\pgfmathsetmacro\len{(#1)+rand*(#2)}}
    -- +(\a:\len pt)
  } -- cycle
}



% C H E M :
\newcommand\CHEMnetwork{

%Nodes
\node [vspecies] (S) {$X_0$};
\node [vspecies, right of = S] (X1) {$X_{1}$};
\node [vspecies, right of = X1] (X2) {$X_{2}$};
\node [vspecies, above right of = X2] (X3) {$X_{3}$};
\node [vspecies, below right of = X2] (X4) {$X_{4}$};
\node [vspecies, right of = X3] (X5) {$X_{5}$};
\node [vspecies, right of = X4] (X6) {$X_{6}$};
%Arows
\draw [<->,thick] (S) --  node {$k_{01}$} (X1) ;
\draw [<->,thick] (X1) --  node {$k_{12}$}(X2);
\draw [<->,thick] (X2) --  node {$k_{23}$}(X3);
\draw [<->,thick] (X2) --  node {$k_{24}$}(X4);
\draw [<->,thick] (X3) --  node {$k_{35}$}(X5) ;
\draw [<->,thick] (X4) --  node {$k_{46}$}(X6);

\draw [<->,thick] (X1) --  node[anchor=north] {$k_{14}$}(X4);
\draw [<->,thick] (S) to[out=40,in=170]  node[anchor=north] {$k_{03}$}(X3);
\draw [<->,thick] (X2) --  node[anchor=north] {$k_{25}$}(X5);
}


\newcommand\MeshLayers{

    % %slanting: production of a set of n 'laminae' to be piled up. N=number of grids.
    % \begin{scope}[
    % 	yshift=170,every node/.append style={
    % 	    yslant=0.5,xslant=-1},yslant=0.5,xslant=-1
    % 	  ]
    %     \fill[white,fill opacity=0.6] (0,0) rectangle (5,5);
    %     \draw[step=10mm, black] (2,2) grid (5,5);
    %     \draw[step=2mm, green] (2,2) grid (3,3);
    %     \draw[black,very thick] (2,2) rectangle (5,5);
    %     \draw[black,dashed] (0,0) rectangle (5,5);
    % \end{scope}
    	
    	

    % \begin{scope}[
    % 	yshift=90,every node/.append style={
    % 	yslant=0.5,xslant=-1},yslant=0.5,xslant=-1
    % 	             ]
    % 	\fill[white,fill opacity=.9] (0,0) rectangle (5,5);
    % 	\draw[step=10mm, black] (1,1) grid (4,4);
    % 	\draw[black,very thick] (1,1) rectangle (4,4);
    % 	\draw[black,dashed] (0,0) rectangle (5,5);
    % \end{scope}
     \begin{scope}[
        yshift=-170,every node/.append style={
        yslant=0.5,xslant=-1},yslant=0.5,xslant=-1
                  ]
        %marking border
        \draw[black,very thick] (0,0) rectangle (5,5);

       	%drawing corners (P1,P2, P3): only 3 points needed to define a plane.
        \draw [fill=blue](0,0) circle (.1) ;
        \draw [fill=blue](0,5) circle (.1);
        \draw [fill=blue](5,0) circle (.1);
        \draw [fill=blue](5,5) circle (.1);

        %drawing bathymetric hypotetic countours on the bottom grid:    	
        \draw [ultra thick,black!70](0,1) parabola bend (2,2) (5,1)  ;
        \draw [dashed] (0,1.5) parabola bend (2.5,2.5) (5,1.5) ;
        \draw [dashed] (0,2) parabola bend (2.7,2.7) (5,2)  ;
        \draw [dashed] (0,2.5) parabola bend (3.5,3.5) (5,2.5)  ;
        \draw [dashed] (0,3.5)  parabola bend (2.75,4.5) (5,3.5);
        \draw [dashed] (0,4)  parabola bend (2.75,4.8) (5,4);
        \draw [dashed] (0,3)  parabola bend (2.75,3.8) (5,3);
        \draw[thick](1,1)node[right]{$\mathsf{Tierra}$};
        \draw[thick](1,4)node[right]{$\mathsf{Mar}$};
    \end{scope} %end of drawing grids

      \begin{scope}[
            yshift=-83,every node/.append style={
            yslant=0.5,xslant=-1},yslant=0.5,xslant=-1
            ]
    	\draw[blue,dashed,fill=blue!10,opacity=0.7] (0,0) rectangle (5,5);
        \draw[blue,thick](1,2.5)node[right]{$\mathsf{Dominio}$};
    \end{scope}  	

    	
    \begin{scope}[
        yshift=0,every node/.append style={
     	yslant=0.5,xslant=-1},yslant=0.5,xslant=-1]
        % opacity to prevent graphical interference
        \fill[white,fill opacity=0.9] (0,0) rectangle (5,5);
        \draw[step=5mm, blue] (0,0) grid (5,5); %defining grids
        %\draw[step=1mm, blue!50,thin] (3,1) grid (4,2);  %Nested Grid
        \draw[black!70,very thick] (0,0) rectangle (5,5);%marking borders
        \fill[blue!70] (0.075,0.075) rectangle (0.425,0.425);
        %Idem as above, for the n-th grid:
    \end{scope}
    
    %putting arrows and labels:
    \node[blue] at (6.0,2.5) {$\}\mathsf{Capa}$};
    \draw[-latex,thick,blue](4.5,0.75)node[right]{$\mathsf{Celda}$}
        to[out=150,in=45](0.,0.25);
    %\draw[-latex,thick,blue](5,3.5)node[right]{$\mathsf{Anidado}$}
    %    to[out=160,in=70] (2.5,3);

    %drawing points on grid's conrners.
    \fill[black,font=\footnotesize]
        (-5,-4.3) node [above] {$(\text{y}_{max},\,\text{x}_{min})$}
        %(-.3,-5.6) node [below] {$P_{2}$}
        (5.5,-4) node [below] {$(\text{y}_{min},\,\text{x}_{max})$};	
}

\newcommand\DrawSistCoordenadas{
   \shade [ball color=themecolor,fill opacity=0.5,earth] (0,0,0) circle [radius=2cm];

  \fill[fill=black] (2,2,2) circle (2pt);

	\draw[-latex, very thick] (0,0,0)--(0,3,0) node[anchor=east]{$x$};
    \draw[-latex, very thick] (0,0,0)--(3,0,0)node[below]{$y$};
    \draw[-latex, very thick] (0,0,0)--(0,0,5)node[below]{$z$};
  \fill[fill=black] (2,2,2) circle (2pt) node[above,anchor=south west]{$x,y,z$};

    \draw[dashed,black!50](2,0,2)--(0,0,0);
    \draw[dashed,black!50](2,2,2)--(2,0,2);
      \draw[-latex,very thick, red!70,] (.75,0,0) arc (0:45:.75)node[anchor=west,midway]{$\phi$};
      \draw[-latex,very thick, red!70,] (0,0,0)--(2,2,2)node[anchor=west,midway,above]{$r$};
      \tdplotsetmaincoords{70}{110}
   \tdplotdrawarc{[-latex,red!70,very thick](0,0,0)}{1}{0}{45}{anchor=north}{$\theta$}
  \fill[fill=black] (2,2,2) circle (2pt) node[below,anchor=north west,red!70]{$\phi,\theta,r$};
}
\newcommand\DrawCoordenadasVerticales{


    %\draw [draw=gray, ->, dashed,thin] (-420, 0) -- (420, 0) node [at end, below] {$x$};
    \draw [-latex,thick,orange] (-4.20,0) -- (-4.20,2) node [midway, left]  {$z$};
    \draw [latex-,thick,orange] ( 4.20,0)  -- (4.20,2) node [midway, right] {$p$};

     \draw [very thick,black!90]   plot [domain=-4.00:4.00, samples=100,name path=A] (\x, {abs(\x*100) >= 180 ? 0 : 1.5*(0.81 - (\x*100/200)^2)});
    \fill [black!80,ground]
      (-4.00,-.2)--(-4.00, 0)--(-1.80,0)
      -- plot[domain=-1.80:1.80,variable=\x] ({\x}, {1.5*(0.81 - (\x*100/200)^2)})
      -- (1.80,0)--(4.00, 0)--(4.00,-.2)
      -- cycle;

     \node[black!80] at (0,0.1) [align=center] {\grotesk Topografía};

  \onslide<1>{}

  \onslide<2>{ \foreach \i in {1,...,6}{
   \draw [very thin, themecolor,dashed,yshift=\i*.3cm] 
     plot [domain=-400:400, samples=100] (\x/100, {0});
   }
   \node at (-4.20,-0.5){\grotesk  Coordenada $z$};
   \node at (-4.20,-1.5){\Large $z$};
   \node at (4.20,-0.5){\grotesk Coordenada $p$};
   \node at (4.20,-1.5){\Large $p$};
   }

  \onslide<3>{ \foreach \i in {.2,.4,...,1.}{
   \draw [very thin, themecolor,dashed]  plot [domain=-4.00:4.00, samples=100] (\x, {abs(\x*100) >= 180 ? 2.*(\i*2. - 2.)/(-2.) : 2.*( (\i*2. - 2.)/(1.5*(0.81 - (\x*100/200)^2) - 2. ) )   });
   }
   \node at (-4.20,-0.5){ \grotesk  Coordenada $\sigma_z$};
   \node at (-4.20,-1.5){\large $\sigma_z = \dfrac{z - z^{top}}{z_{surf}-z^{top}}$};
   \node at (4.20,-0.5){\grotesk  Coordenada $\sigma_p$};
   \node at (4.20,-1.5){\large $\sigma_z = \dfrac{p^{top}-p}{p^{top}-p_{surf}}$};

   }
   \onslide<4>{\node at (0,-1.5){{\large {\grotesk Coordenadas isentrópicas!!} \\ (= entropía)}};}

}

\newcommand\DrawStaggering{


\begin{scope}[xshift=0cm]
    \draw[very thick,blue!50] (-1,-1) rectangle (1,1);
        \draw[fill=blue] ( 0, 0)   circle (0.05) node[anchor=north] {$(i,j)$};
        \draw[fill=blue] (-1,-1) circle (0.05) node {$(u,v,\rho)$};
        \draw[fill=blue](-1, 1) circle (0.05) node {$(u,v,\rho)$};
        \draw[fill=blue] ( 1, 1) circle (0.05) node {$(u,v,\rho)$};
        \draw[fill=blue] ( 1,-1) circle (0.05) node {$(u,v,\rho)$};
\end{scope}

\begin{scope}[xshift=3cm]
    \draw[very thick,blue!50] (-1,-1) rectangle (1,1);
        \draw[fill=blue] ( 0, 0) circle (0.05) node[anchor=north] {$(i,j)$} node[anchor=south]{$(u,v)$};
        \draw[fill=blue] (-1,-1) circle (0.05) node {$\rho$};
        \draw[fill=blue] (-1, 1) circle (0.05) node {$\rho$};
        \draw[fill=blue] ( 1, 1) circle (0.05) node {$\rho$};
        \draw[fill=blue] ( 1,-1) circle (0.05) node {$\rho$};
\end{scope}

\begin{scope}[xshift=6cm]
    \draw[very thick,blue!50] (-1,-1) rectangle (1,1);
        \draw[fill=blue] ( 0, 0)   circle (0.05) node[anchor=north] {$(i,j)$};
        \draw[fill=blue] (-1,-1) circle (0.05) node {$\rho$};
        \draw[fill=blue] (-1, 0) circle (0.05) node {$v$};
        \draw[fill=blue] (-1, 1) circle (0.05) node {$\rho$};
        \draw[fill=blue]( 0, 1) circle (0.05) node {$u$};
        \draw[fill=blue] ( 1, 1) circle (0.05) node {$\rho$};
        \draw[fill=blue]( 1, 0) circle (0.05) node {$v$};
        \draw[fill=blue] ( 1,-1) circle (0.05) node {$\rho$};
        \draw[fill=blue]( 0,-1) circle (0.05) node {$u$};
\end{scope}

\begin{scope}[xshift=9cm]
    \draw[very thick,blue!50] (-1,-1) rectangle (1,1);
        \draw[fill=blue]  ( 0, 0)   circle (0.05) node[anchor=north] {$(i,j)$};
        \draw[fill=blue]  (-1,-1) circle (0.05) node {$\rho$};
        \draw[fill=blue]  (-1, 0) circle (0.05) node {$u$};
        \draw[fill=blue] (-1, 1) circle (0.05) node {$\rho$};
        \draw[fill=blue] ( 0, 1) circle (0.05) node {$v$};
        \draw[fill=blue]  ( 1, 1) circle (0.05) node {$\rho$};
        \draw[fill=blue] ( 1, 0) circle (0.05) node {$u$};
        \draw[fill=blue] ( 1,-1) circle (0.05) node {$\rho$};
        \draw[fill=blue]  ( 0,-1) circle (0.05) node {$v$};
\end{scope}

\begin{scope}[xshift=12cm]
    \draw[very thick,blue!50] (-1,-1) rectangle (1,1);
        \draw[fill=blue] ( 0, 0)   circle (0.05) node[anchor=north] {$(i,j)$};
        \draw[fill=blue] (-1,-1) circle (0.05) node {$(u,v)$};
        \draw[fill=blue] (-1, 0) circle (0.05) node {$\rho$};
        \draw[fill=blue] (-1, 1) circle (0.05) node {$(u,v)$};
        \draw[fill=blue] ( 0, 1) circle (0.05) node {$\rho$};
        \draw[fill=blue] ( 1, 1) circle (0.05) node {$(u,v)$};
        \draw[fill=blue] ( 1, 0) circle (0.05) node {$\rho$};
        \draw[fill=blue] ( 1,-1) circle (0.05) node {$(u,v)$};
        \draw[fill=blue] ( 0,-1) circle (0.05) node {$\rho$};
\end{scope}

}




\newcommand\DrawLagrangeParticulas{
 \begin{tikzpicture}
\begin{scope}[xshift=-4cm]

\foreach \x in {0.,.5,...,2.0} {
    \foreach \y in {0.,0.5,..., 2.0} {
        %\draw[->,thick,blue!50]  (\x, \y)--++(0.15*\x,0.15*\x*\y-0.08*\x);
        \fill[black!75] (\x, \y) circle[radius=1.5pt];
    }
}
\end{scope}

%\draw (0, 0) grid (2.5, 2.5);
\foreach \x in {0.,.5,...,2.0} {
    \foreach \y in {0.,0.5,..., 2.0} {
        \draw[->,thick,blue!70]  (\x, \y)--++(0.1*\x+0.15,0.15*\x*\y*\y-0.15*\x);
        \fill[black!75] (\x,\y) circle[radius=1.5pt];
    }
}
\begin{scope}[xshift=4cm]

\foreach \x in {0.,.5,...,2.0} {
    \foreach \y in {0.,0.5,..., 2.0} {
        \draw[->,thick,blue!20]  (\x, \y)--++(0.1*\x+0.15,0.15*\x*\y*\y-0.15*\x);
        \fill[black!75] (\x, \y)++(0.1*\x+0.15,0.15*\x*\y*\y-0.15*\x) circle[radius=1.5pt];
    }
}
\end{scope}

\end{tikzpicture}
}







\usepackage{pgfplots}
\pgfmathdeclarefunction{gauss}{2}{%
  \pgfmathparse{1/(#2*sqrt(2*pi))*exp(-((x-#1)^2)/(2*#2^2))}%
}

\pgfmathdeclarefunction{LogNormal}{2}{%
  \pgfmathparse{1/(#2*sqrt(2*pi))*exp(-((log(x-#1))^2)/(2*#2^2))}%
}

\pgfmathdeclarefunction{gaussPluma}{3}{%
  \pgfmathparse{1/sqrt(2*pi*(2*#2*x/#1))*exp(-(1/(2*(2*#2*x/#1)^2))-x*#3/#1)}%
}





\newcommand{\EmisPict}{
\begin{tikzpicture}[scale=1.2,inside/.style={draw opacity=.6,thick,dashed},
                        %outside/.style={thick},
                        every node/.style={font=\small}]
        %lateral sides                
        \draw[inside](0.5,0.5,0.5)--(0.5,-0.5,0.5)--(0.5,-0.5,-0.5)--(0.5,0.5,-0.5)--(0.5,0.5,0.5);
        \draw[inside](-0.5,-0.5,-0.5)--(-0.5,0.5,-0.5)--(-0.5,0.5,0.5)--(-0.5,-0.5,0.5)--(-0.5,-0.5,-0.5);
        %--
        \draw[inside](-0.5,-0.5,-0.5)--(-0.5,0.5,-0.5)--(-0.5,0.5,0.5)--(-0.5,-0.5,0.5)--(-0.5,-0.5,-0.5);
        \draw[inside,fill=themecolor!10](-0.5,-0.5,-0.5)--(0.5,-0.5,-0.5)--(0.5,-0.5,0.5)--(-0.5,-0.5,0.5)--(-0.5,-0.5,-0.5);
        \draw[inside](-0.5,-0.5,-0.5)--(0.5,-0.5,-0.5)--(0.5,0.5,-0.5)--(-0.5,0.5,-0.5)--(-0.5,-0.5,-0.5);
        \draw[inside](0.5,0.5,0.5)--(-0.5,0.5,0.5)--(-0.5,0.5,-0.5)--(0.5,0.5,-0.5)--(0.5,0.5,0.5);
        \draw[inside](0.5,0.5,0.5)--(-0.5,0.5,0.5)--(-0.5,-0.5,0.5)--(0.5,-0.5,0.5)--(0.5,0.5,0.5);
        \draw[dashed](-0.5,-0.5,-0.5)--(0.5,-0.5,-0.5)--(0.5,-0.5,0.5)--(-0.5,-0.5,0.5)--(-0.5,-0.5,-0.5);
        \draw[dashed](-0.5,-0.5,-0.5)--(0.5,-0.5,-0.5)--(0.5,0.5,-0.5)--(-0.5,0.5,-0.5)--(-0.5,-0.5,-0.5);      
        %% inside frame - this shouldn't be necessary...
    %Emision:
 	\draw[-stealth,line width=0.5mm](0.0,-0.5,0)--(0,0.15,0) node[anchor=west,midway] {$E$};
 	 \axisDDD{0.5}{-2cm}{-0.5cm}
\end{tikzpicture}
}
\newcommand{\AdvectPict}{
\begin{tikzpicture}[scale=1.2,inside/.style={draw opacity=.5,thick,dashed},
                        %outside/.style={thick},
                        every node/.style={font=\small}]
        %lateral sides                
        \draw[inside,fill=red!10](0.5,0.5,0.5)--(0.5,-0.5,0.5)--(0.5,-0.5,-0.5)--(0.5,0.5,-0.5)--(0.5,0.5,0.5);
        \draw[dashed,fill=blue!10](-0.5,-0.5,-0.5)--(-0.5,0.5,-0.5)--(-0.5,0.5,0.5)--(-0.5,-0.5,0.5)--(-0.5,-0.5,-0.5);
        %--
        \draw[inside](-0.5,-0.5,-0.5)--(-0.5,0.5,-0.5)--(-0.5,0.5,0.5)--(-0.5,-0.5,0.5)--(-0.5,-0.5,-0.5);
        \draw[inside](-0.5,-0.5,-0.5)--(0.5,-0.5,-0.5)--(0.5,-0.5,0.5)--(-0.5,-0.5,0.5)--(-0.5,-0.5,-0.5);
        \draw[inside](-0.5,-0.5,-0.5)--(0.5,-0.5,-0.5)--(0.5,0.5,-0.5)--(-0.5,0.5,-0.5)--(-0.5,-0.5,-0.5);
        \draw[inside](0.5,0.5,0.5)--(-0.5,0.5,0.5)--(-0.5,0.5,-0.5)--(0.5,0.5,-0.5)--(0.5,0.5,0.5);
        \draw[inside](0.5,0.5,0.5)--(-0.5,0.5,0.5)--(-0.5,-0.5,0.5)--(0.5,-0.5,0.5)--(0.5,0.5,0.5);
        \draw[dashed](-0.5,-0.5,-0.5)--(0.5,-0.5,-0.5)--(0.5,-0.5,0.5)--(-0.5,-0.5,0.5)--(-0.5,-0.5,-0.5);
        \draw[dashed](-0.5,-0.5,-0.5)--(0.5,-0.5,-0.5)--(0.5,0.5,-0.5)--(-0.5,0.5,-0.5)--(-0.5,-0.5,-0.5);      
%VIENTO:
 	\draw[-stealth,line width=0.5mm](-1.5,0,0)--(-0.5,0,0) node[anchor=north,midway] {$Q_{1}C_{1}$};
 	\draw[-stealth,line width=0.5mm](0.5,0,0)--(1.5,0,0) node[anchor=north,midway] {$Q_{2}C_{2}$};
    %\node[]at(-1.2,0,0){$C_{1}$};
	%\node[]at(0,0,0){$C_{2}$};
%\pause
    \axisDDD{0.5}{-2cm}{-0.5cm}
\end{tikzpicture}
}

\newcommand{\DiffPict}{
\begin{tikzpicture}[scale=1.2,inside/.style={draw opacity=.5,thick,dashed},%outside/.style={thick},
    every node/.style={font=\small}]
        %lateral sides                
        \draw[inside,fill=red!10](0.5,0.5,0.5)--(0.5,-0.5,0.5)--(0.5,-0.5,-0.5)--(0.5,0.5,-0.5)--(0.5,0.5,0.5);
        \draw[dashed,fill=blue!10](-0.5,-0.5,-0.5)--(-0.5,0.5,-0.5)--(-0.5,0.5,0.5)--(-0.5,-0.5,0.5)--(-0.5,-0.5,-0.5);
        %--
        \draw[inside](-0.5,-0.5,-0.5)--(-0.5,0.5,-0.5)--(-0.5,0.5,0.5)--(-0.5,-0.5,0.5)--(-0.5,-0.5,-0.5);
        \draw[inside](-0.5,-0.5,-0.5)--(0.5,-0.5,-0.5)--(0.5,-0.5,0.5)--(-0.5,-0.5,0.5)--(-0.5,-0.5,-0.5);
        \draw[inside](-0.5,-0.5,-0.5)--(0.5,-0.5,-0.5)--(0.5,0.5,-0.5)--(-0.5,0.5,-0.5)--(-0.5,-0.5,-0.5);
        \draw[inside](0.5,0.5,0.5)--(-0.5,0.5,0.5)--(-0.5,0.5,-0.5)--(0.5,0.5,-0.5)--(0.5,0.5,0.5);
        \draw[inside](0.5,0.5,0.5)--(-0.5,0.5,0.5)--(-0.5,-0.5,0.5)--(0.5,-0.5,0.5)--(0.5,0.5,0.5);
        \draw[dashed](-0.5,-0.5,-0.5)--(0.5,-0.5,-0.5)--(0.5,-0.5,0.5)--(-0.5,-0.5,0.5)--(-0.5,-0.5,-0.5);
        \draw[dashed](-0.5,-0.5,-0.5)--(0.5,-0.5,-0.5)--(0.5,0.5,-0.5)--(-0.5,0.5,-0.5)--(-0.5,-0.5,-0.5);      
    %VIENTO:
 	\draw[-stealth,line width=0.5mm](-1.5,0,0)--(-0.5,0,0) node[anchor=north,midway] {$J_{1}A$};
 	\draw[-stealth,line width=0.5mm](0.5,0,0)--(1.5,0,0) node[anchor=north,midway] {$J_{2}A$};
 	
 	\axisDDD{0.5}{-2cm}{-0.5cm}
\end{tikzpicture}
}

\newcommand{\QuimPict}{
\begin{tikzpicture}[scale=1.2,inside/.style={draw opacity=.5,thick,dashed},%outside/.style={thick},
    every node/.style={font=\small}]
        %lateral sides                
        \draw[inside,fill=red!10](0.5,0.5,0.5)--(0.5,-0.5,0.5)--(0.5,-0.5,-0.5)--(0.5,0.5,-0.5)--(0.5,0.5,0.5);
        \draw[dashed,fill=red!10](-0.5,-0.5,-0.5)--(-0.5,0.5,-0.5)--(-0.5,0.5,0.5)--(-0.5,-0.5,0.5)--(-0.5,-0.5,-0.5);
        %--
        \draw[inside,fill=red!10](-0.5,-0.5,-0.5)--(-0.5,0.5,-0.5)--(-0.5,0.5,0.5)--(-0.5,-0.5,0.5)--(-0.5,-0.5,-0.5);
        \draw[inside,fill=red!10](-0.5,-0.5,-0.5)--(0.5,-0.5,-0.5)--(0.5,-0.5,0.5)--(-0.5,-0.5,0.5)--(-0.5,-0.5,-0.5);
        \draw[inside,fill=red!10](-0.5,-0.5,-0.5)--(0.5,-0.5,-0.5)--(0.5,0.5,-0.5)--(-0.5,0.5,-0.5)--(-0.5,-0.5,-0.5);
        \draw[inside,fill=red!10](0.5,0.5,0.5)--(-0.5,0.5,0.5)--(-0.5,0.5,-0.5)--(0.5,0.5,-0.5)--(0.5,0.5,0.5);
        \draw[inside,fill=red!10](0.5,0.5,0.5)--(-0.5,0.5,0.5)--(-0.5,-0.5,0.5)--(0.5,-0.5,0.5)--(0.5,0.5,0.5);
        \draw[dashed](-0.5,-0.5,-0.5)--(0.5,-0.5,-0.5)--(0.5,-0.5,0.5)--(-0.5,-0.5,0.5)--(-0.5,-0.5,-0.5);
        \draw[dashed](-0.5,-0.5,-0.5)--(0.5,-0.5,-0.5)--(0.5,0.5,-0.5)--(-0.5,0.5,-0.5)--(-0.5,-0.5,-0.5);      
        
 	    \axisDDD{0.5}{-2cm}{-0.5cm}
\end{tikzpicture}
}

%%tikz pictures:
%\newcommand{\EmisPict}{
%\begin{tikzpicture}[scale=1.5,inside/.style={draw opacity=.6,thick,dashed},
%                        %outside/.style={thick},
%                        every node/.style={font=\small}]
%        %lateral sides                
%        \draw[inside](0.5,0.5,0.5)--(0.5,-0.5,0.5)--(0.5,-0.5,-0.5)--(0.5,0.5,-0.5)--(0.5,0.5,0.5);
%        \draw[inside](-0.5,-0.5,-0.5)--(-0.5,0.5,-0.5)--(-0.5,0.5,0.5)--(-0.5,-0.5,0.5)--(-0.5,-0.5,-0.5);
%        %--
%        \draw[inside](-0.5,-0.5,-0.5)--(-0.5,0.5,-0.5)--(-0.5,0.5,0.5)--(-0.5,-0.5,0.5)--(-0.5,-0.5,-0.5);
%        \draw[inside,fill=themecolor!10](-0.5,-0.5,-0.5)--(0.5,-0.5,-0.5)--(0.5,-0.5,0.5)--(-0.5,-0.5,0.5)--(-0.5,-0.5,-0.5);
%        \draw[inside](-0.5,-0.5,-0.5)--(0.5,-0.5,-0.5)--(0.5,0.5,-0.5)--(-0.5,0.5,-0.5)--(-0.5,-0.5,-0.5);
%        \draw[inside](0.5,0.5,0.5)--(-0.5,0.5,0.5)--(-0.5,0.5,-0.5)--(0.5,0.5,-0.5)--(0.5,0.5,0.5);
%        \draw[inside](0.5,0.5,0.5)--(-0.5,0.5,0.5)--(-0.5,-0.5,0.5)--(0.5,-0.5,0.5)--(0.5,0.5,0.5);
%        \draw[dashed](-0.5,-0.5,-0.5)--(0.5,-0.5,-0.5)--(0.5,-0.5,0.5)--(-0.5,-0.5,0.5)--(-0.5,-0.5,-0.5);
%        \draw[dashed](-0.5,-0.5,-0.5)--(0.5,-0.5,-0.5)--(0.5,0.5,-0.5)--(-0.5,0.5,-0.5)--(-0.5,-0.5,-0.5);      
%        %% inside frame - this shouldn't be necessary...
%    %Emision:
% 	\draw[-stealth,line width=0.5mm](0.0,-0.5,0)--(0,0.15,0) node[anchor=west,midway] {$E$};
% 	 \axisDDD{0.5}{-2cm}{-0.5cm}
%\end{tikzpicture}
%}
%
%\newcommand{\AdvectPict}{
%\begin{tikzpicture}[scale=1.5,inside/.style={draw opacity=.5,thick,dashed},
%                        %outside/.style={thick},
%                        every node/.style={font=\small}]
%        %lateral sides                
%        \draw[inside,fill=red!10](0.5,0.5,0.5)--(0.5,-0.5,0.5)--(0.5,-0.5,-0.5)--(0.5,0.5,-0.5)--(0.5,0.5,0.5);
%        \draw[dashed,fill=blue!10](-0.5,-0.5,-0.5)--(-0.5,0.5,-0.5)--(-0.5,0.5,0.5)--(-0.5,-0.5,0.5)--(-0.5,-0.5,-0.5);
%        %--
%        \draw[inside](-0.5,-0.5,-0.5)--(-0.5,0.5,-0.5)--(-0.5,0.5,0.5)--(-0.5,-0.5,0.5)--(-0.5,-0.5,-0.5);
%        \draw[inside](-0.5,-0.5,-0.5)--(0.5,-0.5,-0.5)--(0.5,-0.5,0.5)--(-0.5,-0.5,0.5)--(-0.5,-0.5,-0.5);
%        \draw[inside](-0.5,-0.5,-0.5)--(0.5,-0.5,-0.5)--(0.5,0.5,-0.5)--(-0.5,0.5,-0.5)--(-0.5,-0.5,-0.5);
%        \draw[inside](0.5,0.5,0.5)--(-0.5,0.5,0.5)--(-0.5,0.5,-0.5)--(0.5,0.5,-0.5)--(0.5,0.5,0.5);
%        \draw[inside](0.5,0.5,0.5)--(-0.5,0.5,0.5)--(-0.5,-0.5,0.5)--(0.5,-0.5,0.5)--(0.5,0.5,0.5);
%        \draw[dashed](-0.5,-0.5,-0.5)--(0.5,-0.5,-0.5)--(0.5,-0.5,0.5)--(-0.5,-0.5,0.5)--(-0.5,-0.5,-0.5);
%        \draw[dashed](-0.5,-0.5,-0.5)--(0.5,-0.5,-0.5)--(0.5,0.5,-0.5)--(-0.5,0.5,-0.5)--(-0.5,-0.5,-0.5);      
%        %% inside frame - this shouldn't be necessary...
%
%
%%VIENTO:
% 	\draw[-stealth,line width=0.5mm](-1.5,0,0)--(-0.5,0,0) node[anchor=north,midway] {$Q_{1}C_{1}$};
% 	\draw[-stealth,line width=0.5mm](0.5,0,0)--(1.5,0,0) node[anchor=north,midway] {$Q_{2}C_{2}$};
%    %\node[]at(-1.2,0,0){$N_{1}$};
%	%\node[]at(0,0,0){$N_{2}$};
%%\pause
%%Caja entrada:
%%\draw[fill=themecolor!10,opacity=0.1](-1,-0.5,-0.5)--(-1,-0.5,0.5)--(-1,0.5,0.5)--(-1,0.5,-0.50)--(-1,-0.5,-0.5);
%%\draw[fill=themecolor!10,opacity=0.1](-1,0.5,-0.5)--(-0.5,0.5,-0.5)--(-0.5,0.5,0.5)--(-1,0.5,0.50)--(-1,0.5,-0.5);
%%\draw[fill=themecolor!10,opacity=0.1](-1,-0.5,0.5)--(-0.5,-0.5,0.5)--(-0.5,0.5,0.5)--(-1,0.5,0.50)--(-1,-0.5,0.5);
%%\pause
%%Caja Salida
%%\draw[fill=themecolor!40,opacity=0.1](0.5,0.5,-0.5)--(1,0.5,-0.5)--(1,0.5,0.5)--(0.5,0.5,0.50)--(0.5,0.5,-0.5);
%%\draw[fill=themecolor!40,opacity=0.1](0.5,-0.5,0.5)--(1,-0.5,0.5)--(1,0.5,0.5)--(0.5,0.5,0.50)--(0.5,-0.5,0.5);
%%\draw[fill=themecolor!40,opacity=0.1](1,-0.5,-0.5)--(1,-0.5,0.5)--(1,0.5,0.5)--(1,0.5,-0.50)--(1,-0.5,-0.5);
%\end{tikzpicture}
%}
%
%\newcommand{\DiffPict}{
%\begin{tikzpicture}[scale=1.5,inside/.style={draw opacity=.5,thick,dashed},%outside/.style={thick},
%    every node/.style={font=\small}]
%        %lateral sides                
%        \draw[inside,fill=red!10](0.5,0.5,0.5)--(0.5,-0.5,0.5)--(0.5,-0.5,-0.5)--(0.5,0.5,-0.5)--(0.5,0.5,0.5);
%        \draw[dashed,fill=blue!10](-0.5,-0.5,-0.5)--(-0.5,0.5,-0.5)--(-0.5,0.5,0.5)--(-0.5,-0.5,0.5)--(-0.5,-0.5,-0.5);
%        %--
%        \draw[inside](-0.5,-0.5,-0.5)--(-0.5,0.5,-0.5)--(-0.5,0.5,0.5)--(-0.5,-0.5,0.5)--(-0.5,-0.5,-0.5);
%        \draw[inside](-0.5,-0.5,-0.5)--(0.5,-0.5,-0.5)--(0.5,-0.5,0.5)--(-0.5,-0.5,0.5)--(-0.5,-0.5,-0.5);
%        \draw[inside](-0.5,-0.5,-0.5)--(0.5,-0.5,-0.5)--(0.5,0.5,-0.5)--(-0.5,0.5,-0.5)--(-0.5,-0.5,-0.5);
%        \draw[inside](0.5,0.5,0.5)--(-0.5,0.5,0.5)--(-0.5,0.5,-0.5)--(0.5,0.5,-0.5)--(0.5,0.5,0.5);
%        \draw[inside](0.5,0.5,0.5)--(-0.5,0.5,0.5)--(-0.5,-0.5,0.5)--(0.5,-0.5,0.5)--(0.5,0.5,0.5);
%        \draw[dashed](-0.5,-0.5,-0.5)--(0.5,-0.5,-0.5)--(0.5,-0.5,0.5)--(-0.5,-0.5,0.5)--(-0.5,-0.5,-0.5);
%        \draw[dashed](-0.5,-0.5,-0.5)--(0.5,-0.5,-0.5)--(0.5,0.5,-0.5)--(-0.5,0.5,-0.5)--(-0.5,-0.5,-0.5);      
%    %VIENTO:
% 	\draw[-stealth,line width=0.5mm](-1.5,0,0)--(-0.5,0,0) node[anchor=north,midway] {$J_{1}A$};
% 	\draw[-stealth,line width=0.5mm](0.5,0,0)--(1.5,0,0) node[anchor=north,midway] {$J_{2}A$};
%\end{tikzpicture}
%}
%
%\newcommand{\QuimPict}{
%\begin{tikzpicture}[scale=1.5,inside/.style={draw opacity=.5,thick,dashed},%outside/.style={thick},
%    every node/.style={font=\small}]
%        %lateral sides                
%        \draw[inside,fill=red!10](0.5,0.5,0.5)--(0.5,-0.5,0.5)--(0.5,-0.5,-0.5)--(0.5,0.5,-0.5)--(0.5,0.5,0.5);
%        \draw[dashed,fill=red!10](-0.5,-0.5,-0.5)--(-0.5,0.5,-0.5)--(-0.5,0.5,0.5)--(-0.5,-0.5,0.5)--(-0.5,-0.5,-0.5);
%        %--
%        \draw[inside,fill=red!10](-0.5,-0.5,-0.5)--(-0.5,0.5,-0.5)--(-0.5,0.5,0.5)--(-0.5,-0.5,0.5)--(-0.5,-0.5,-0.5);
%        \draw[inside,fill=red!10](-0.5,-0.5,-0.5)--(0.5,-0.5,-0.5)--(0.5,-0.5,0.5)--(-0.5,-0.5,0.5)--(-0.5,-0.5,-0.5);
%        \draw[inside,fill=red!10](-0.5,-0.5,-0.5)--(0.5,-0.5,-0.5)--(0.5,0.5,-0.5)--(-0.5,0.5,-0.5)--(-0.5,-0.5,-0.5);
%        \draw[inside,fill=red!10](0.5,0.5,0.5)--(-0.5,0.5,0.5)--(-0.5,0.5,-0.5)--(0.5,0.5,-0.5)--(0.5,0.5,0.5);
%        \draw[inside,fill=red!10](0.5,0.5,0.5)--(-0.5,0.5,0.5)--(-0.5,-0.5,0.5)--(0.5,-0.5,0.5)--(0.5,0.5,0.5);
%        \draw[dashed](-0.5,-0.5,-0.5)--(0.5,-0.5,-0.5)--(0.5,-0.5,0.5)--(-0.5,-0.5,0.5)--(-0.5,-0.5,-0.5);
%        \draw[dashed](-0.5,-0.5,-0.5)--(0.5,-0.5,-0.5)--(0.5,0.5,-0.5)--(-0.5,0.5,-0.5)--(-0.5,-0.5,-0.5);      
%\end{tikzpicture}
%}
%
\newcommand{\CicloDiurnoPBL}[0]{


       \fill[blue!5] (0,2.0) rectangle(24,2.9);
        \fill[blue!20,shading = axis, left color=blue!80, right color=blue!10,shading angle= 70] (0,0) rectangle(12,2);
        \fill[blue!20,shading = axis, left color=blue!80, right color=blue!10,shading angle=-70] (12,0) rectangle(24,2);
        \fill[white] (0,-0.2) rectangle(24,0);
        \draw[-latex,thick,black!80] (0,0)--(0,3)node[anchor=east]{$z$ (km) };
    
        \piso{0}{24}{0}
        \foreach \i in {0,6,...,24} \draw[] (\i,-0.2) --(\i,-0.4)  node[anchor=north]{\i};
        \foreach \i in {0,1,2}    \draw[] (-0.0,\i) --(-0.3,\i)  node[anchor=east ]{\i};
        %stable layer
        \draw[dotted, thick] (0,1)--(8,1);
        \draw[dotted, thick] (18,0) to [out=40,in=180,looseness=0.9] (24,1);
         %capa de inversión
         \draw[thin,black!80,fill=blue!80,decorate] (0,2.1) to[out=3,in=182] (10,2.1) -- (10,1.7) to[in=3,out=182] (0,1.7);
         \draw[thin,black!80,fill=blue!80] (6,0) to [out=15,in=180] (10,2.1)to[out=4,in=182] (24,2.1) -- (24,1.7)to[in=4,out=182](10.2,1.7) to [in=05,out=180] (6.1,0) ;
        
        %surface layer
        \draw[dashed, orange!80] (0,0.2) -- (24,0.2) node[anchor=west]{CS};
        
        %separacion dia noche
        \draw[thick,dashed](18,0)--(18,2);
        \draw[thick,dashed]( 6,0)--( 6,2);
       
        %labels
        \node at (12,1){\Large CM};
        \node at (21,0.5){\Large CE};
        \node at (23,1.3){\Large CR};
        
        \node at (3,0.5){\Large CE};
        \node at (3,1.3){\Large CR};
        
        \node[white,rotate=-2] at ( 3,2.0){\small Inversión};
        \node[white] at (21,1.90){\small Inversión};
        \node[white,rotate=3] at (12,1.97){\small Capa de arrastre};
    
        \node at (12,2.5){\large Atmosfera libre};
        
        \draw[latex-] (6,2.1)-- (6,2.5) node[anchor=south]{\small (Amanece)};
        \draw[latex-](18,2.1)--(18,2.5) node[anchor=south]{\small (Anochece)};
}





\newcommand\ejesXZ[3]{
    \draw[thick,-latex] (#1,#2) --++ (#3,0) node[right] {x};
    \draw[thick,-latex] (#1,#2) --++ (0,#3) node[above] {z};
}

\newcommand\piso[3]{
 %piso
 \fill[ground] (#1,{#3-0.2})rectangle(#2,#3);
 \draw [very thick] (#1,#3)--(#2,#3);
v}

\newcommand\chimenea[4]{ %x y  W H
 %pluma real 
 %\node(chim1)[fill=black!60,minimum height=1.25cm] at (-3,0.6){}; 
 \draw[fill=black!60](#1,#2) rectangle ++(#3,#4);
   
}

\newcommand\pluma[3]{
 \fill[red!20,opacity=0.5,yshift=0.5cm] (chim1.north) -- (4,2) -- (4,-1) -- (chim1.north);
}
% \draw[| latex-latex] (-3.2, 1.2)--(-3.2,0) node[midway, anchor=east]{$H$};
    


\pgfdeclarepatternformonly{my crosshatch dots}{\pgfqpoint{-1pt}{-1pt}}{\pgfqpoint{5pt}{5pt}}{\pgfqpoint{6pt}{6pt}}%
{
    \pgfpathcircle{\pgfqpoint{3pt}{3pt}}{.9pt}
    \pgfpathcircle{\pgfqpoint{3pt}{3pt}}{.9pt}
    \pgfusepath{fill}
}


































