\subtitle{Olores}
\begin{frame}
  \titlepage
\end{frame}
%===========================================================================================
\section{Fundamentos del olor}

\begin{frame}{Compuestos responsables de olores}

Los compuestos responsables de malos olores usualmente se generan por actividad microbiológica en ambientes pobres de oxigeno (reductores).

Algunos compuestos típicamente asociados a malos olores son:
\begin{itemize}
    \item Compuestos orgánicos volátiles (COVs, por ejemplo Etil Mercaptano)
    \item Compuestos reducidos de Azufre (por ejemplo: H2S)
    \item Compuestos reducidos de Nitrogeno (por ejemplo: NH3)
\end{itemize}       
\end{frame}

\begin{frame}{Físico-química del biogas}


   % Formacion de biogas

   % Curvas de CH3, CO2, H2, etc.

   % Condiciones reductoras

\end{frame}



%===========================================================================================
\section{Marco regulatorio y directivas internacionales}
\begin{frame}{Marco regulatorio}
 
    
Regulación y directrices:

   \begin{itemize}
       \item \alert{U.S.A}    \textbf{Clean Air Act} (USEPA)
       \item \alert{Europa}   Norma UNE \textbf{EN 13725}
       \item \alert{Nacional} \textbf{DECRETO 1.074/2018} (Anexo IV).  
   \end{itemize}    
\end{frame}

    
\begin{frame}{Niveles guía / umbrales}
    \begin{table}[]
        \centering
        \begin{tabular}{| l | c |}\hline
                          & Dec. 1074/18           \\
             Contaminante & Umbral de olor (ppmV)  \\\hline
             ... & ...    \\
             Amoniaco (NH3)             &   46.8  \\
             Etil Mercaptano (H3C-CH2-SH)           &   0.0004-0.001  \\ 
             Sulfuro de Hidrógeno (H2S) &   0.005  \\
             ... & ...    \\\hline
        \end{tabular}
        \caption{Tabla III. Umbrales de Olor e Irritación.}
    \end{table}
\end{frame}



%===========================================================================================
\section{Herramientas para el diagnóstico}

\begin{frame}{Herramientas para el diagnóstico}

Vamos a revisar primero que herramientas tenemos para cuantificar el impacto posible que se puede estar produciendo en los alrededores.

\begin{itemize}
    \item Técnicas de \alert{Mediciones de olores}
    \item Estimación de \alert{Emisiones} 
    \item \alert{Modelos de dispersión de olores} 
    
\end{itemize}    
\end{frame}

\begin{frame}{Mediciones}

Para medir \textbf{concentraciones}:
\begin{itemize}
    \item Fencline monitoring: Sensores MOS, Sensores especificos de NH3 y H2S
    \item Dynamic Olfatometry  (Humanos)
    \item Instrumental Odour Monitoring Systems (IOMS)
    \item Narices electrónicas  (Nasal Ranger)   
\end{itemize}    

Para medir \textbf{emisiones}, Utilizan algun compartimiento cerrado de algún tipo (campana, camara de flujo, tunel de viento, etc.), y la emision resulta de el producto entre concenentración y caudal de salida del instrumento.


\end{frame}    


\begin{frame}{Métodos para estimar Emisiones}

Estimación de emisiones.
\begin{itemize}
    \item Tracer methods
    \item Micrometeorological methods
    \item AP-42 (Landgem)
\end{itemize}    
\end{frame}    


\begin{frame}{Modelos de dispersión de olores}

Estimación de emisiones.
\begin{itemize}
    \item AERMOD
    \item Calpuff
\end{itemize}    
\end{frame}    


%===========================================================================================
\section{Estratégias de manejo de las emisiones}


\begin{frame}{Descripción del transporte}

\begin{itemize}
   \item Seleccion de resiudos, disponerlo de forma diferenciada: Ejemplo: NO mezclar construccion con organicos. Reduccion de S del yeso.
   \item Frente de descarga más chico posible, luego de descarga cobertura intermedia, impermeabilizacion de superficie lo más rapido. 
   \item Buen sistema de captación de gas.
   \item Coberturas intermedias "alternativas" (no solo suelo). Por ejemplo: Espumas, chips orgánicos.
   \item Cobertura de relleno: bio-cobertura (actividad orgánica), cultivo.
   \item Mecanísmo activos de control de olores. Venitiladores, spray, nieblas.
   \item Piletas de lixiviado no deben estar tapadas (generan olor).
\end{itemize}

    
\end{frame}


