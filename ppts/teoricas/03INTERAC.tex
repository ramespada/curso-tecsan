\subtitle{Interacción de la pluma con el entorno}
 \begin{frame}{}
     \maketitle
 \end{frame}



\begin{frame}{Ascenso de pluma}
Hay dos tipos de procesos que generan el ascenso de una pluma:
\begin{itemize}
    \item Ascenso mecánico: por transferencia de momentum.
    \item Flotación/Empuje (bouyancy): debido a la diferencia de temperaturas entre el efluente y la atmósfera.
\end{itemize}


Parámetro de flujo:
    $$ F_m = (\frac{\overline{\rho_s}}{\rho}) r^2 \overline{w}^2   \qquad F_b = (1-\frac{\overline{\rho_s}}{\rho})g r^2 \overline{w}^2  $$
    
Ascenso:
    $$ \Delta z = 2 \bigg( \dfrac{F_m x}{\overline{u}^2} \bigg)^{1/3} \qquad    \Delta z = 1.6 \bigg( \dfrac{F_b x^2}{\overline{u}^3}  \bigg)^{1/3}    $$
    
    $$ \Delta h = 2.4^{1/4} \bigg( \dfrac{F_m}{s}\bigg)^{1/4}  \qquad  \Delta h = 5.3^{1/4} \bigg( \dfrac{F_b^2}{s^3} \bigg)^{1/8} $$
    
\end{frame}


% \begin{frame}{Velocidad crítica}
%     La velocidad de la pluma afecta las concentraciones de dos formas antagónicas:
%     \begin{itemize}
%         \item Disminuye las concnentraciones ya que estira la pluma.
%         \item Aumenta las concentraciones en superficie por que decrece el acenso de la pluma.
%     \end{itemize}
    
%     El balance entre estos dos procesos hace que exista un valor máximo de concentraciones a una velocidad dada, conocida como "velocidad crítica".
    
    
% \end{frame}

\begin{frame}{Downwash}
    
    Ocurre por interferencia de la pluma con edificios aledaños.
    La pluma es atrapada por sombras turbulentas que se generan en las caras opuestas al viento de edificios.
    
    $$ \Delta h = 2 d_{s} ( \dfrac{w_s}{u} - 1.5 ) $$
    
\end{frame}


\begin{frame}{Downwash}{Deflexión de la pluma}
\begin{itemize}
    \item (A) Upwind, no afectado por edificios, $dz/dx=0$
    \item (B) Upwind, afectado por edificio:
    $$ \dfrac{dz}{dx}= \dfrac{2(H_R-H)(x+R)}{R^2}$$
    \item (C) Downwind, antes de cabidad
    $$ \dfrac{dz}{dx}= \dfrac{-4(H_R-H)(2x+R-1)}{R}$$

    \item (D) Downwind, sobre la cabidad
    $$ \dfrac{dz}{dx}= \dfrac{(H_R-H)(R-2x)}{(L+L_R - R/2)^2}(\dfrac{z}{H})^{0.3}$$

    \item (E)  Downwind, despues de la cabidad
    $$ \dfrac{dz}{dx}= \dfrac{-(H_R-H)(L+L_R)}{x(L+L_R - R/2)}(\dfrac{z}{H})^{0.3}$$

    

\end{itemize}{}

\end{frame}


\begin{frame}{Deposicion}{Seca}
    
    $$F = - \nu_d c $$
    donde:
    $$\dfrac{1}{\nu_d} = r_T = r_a + r_b + r_c $$
    
    
    Ley de Stokes:
    $$ \nu_s = \dfrac{\rho d^2 g}{18\mu}$$
\end{frame}


\begin{frame}{Deposcion}{Humeda}
    
    
    $$\Lambda = \lambda \dfrac{R}{R_1} $$
    
    
    $$\exp \bigg[ - \Lambda t \bigg] $$
\end{frame}


\begin{frame}{Interaccion con el terreno}
    El perfil de vientos en terrenos densamente poblados de obstaculos se puede estimar:\footnote{}
    $$ u = \frac{u_*}{k} \ln{}\dfrac{z-d}{z_0}$$
\end{frame}